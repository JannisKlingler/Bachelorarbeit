\documentclass[11pt,titlepage]{article}
\usepackage{amsmath,amssymb,amstext,mathtools,amsthm}
\usepackage{xcolor}
\usepackage[utf8]{inputenc}
\usepackage[ngerman]{babel}
%\usepackage[paper=a4paper,left=25mm,right=25mm,top=25mm,bottom=25mm]{geometry}
\usepackage[twoside,lmargin=3.5cm,rmargin=2.5cm]{geometry}
\usepackage{hyperref}
\hypersetup{bookmarksnumbered}

\usepackage{dsfont}
%\usepackage{xfrac}
\usepackage{tikz}
\usepackage{graphicx}
\usepackage{bigints}
\usepackage{bibgerm}
\usepackage[onehalfspacing]{setspace}

\usetikzlibrary{positioning}
%\usetikzlibrary{arrows}

\newcommand{\setN}{\mathbb{N}}
\newcommand{\setZ}{\mathbb{Z}}
\newcommand{\setQ}{\mathbb{Q}}
\newcommand{\setR}{\mathbb{R}}
\newcommand{\setC}{\mathbb{C}}
\newcommand{\setH}{\mathbb{H}}
\newcommand{\setI}{\mathbb{I}}
\newcommand{\abs}[1]{{\left| #1 \right|}}

\theoremstyle{definition}
\newtheorem{theorem}{Satz}[section]
\newtheorem{corollary}[theorem]{Folgerung}
\newtheorem{proposition}[theorem]{Proposition}
\newtheorem{lemma}[theorem]{Lemma}
\newtheorem{definition}[theorem]{Definition}
\newtheorem{example}[theorem]{Beispiel}
\newtheorem*{axiom}{Axiom}
\newtheorem{remark}[theorem]{Bemerkung}

\theoremstyle{remark}
\newtheorem*{repetition}{Wiederholung}
\newtheorem*{remind}{Erinnerung}

\begin{document}
	\section{Ergänzungsgleichheit von Polytopen}
	
	Wir wollen uns nun dem zweiten Teil von Hilberts Problem widmen und zwar der 
	Ergänzungsgleichheit von Polytopen. Wir wollen in diesem Kapitel zeigen, dass 
	Zerlegungsgleichheit und Ergänzungsgleichheit gleichbedeutend sind. Der 
	Beweis hierzu ist nicht gerade trivial, weswegen wir vorweg noch einige 
	Kenntnisse über Polytope benötigen. Das Kapitel richtet sich nach 
	\cite{Hadwiger}. 
	Im Folgenden meinen wir mit Polytopen wieder $d$-Polytope. 
	
	\subsection{Ein wenig Polyedertheorie}
	
	Wir wollen uns zunächst überlegen, wie wir Polytope vervielfältigen, sowie 
	über eine Dilation stauchen und strecken können. 
	
	\begin{itemize}
		\item \textsl{Dilation:} Unter einer Dilation mit einem Faktor $\lambda>0$ 
		verstehen wir eine Streckung bzw. Stauchung aller Punkte 
		im Koordinatensystem. Ein Punkt $p\in\setR^d$ wird also auf 
		sein skalares Vielfaches $\lambda\cdot p$ abgebildet. Im Zusammenhang mit 
		Polytopen erhalten wir mit dem Faktor $\lambda$ das dilatierte Polytop 
		$\lambda P$. \\
		Wir bemerken, dass wenn wir diese Abbildungen auf Polytope anwenden wir wieder Polytope erhalten, da Halbräume auf Halbräume abgebildet werden.
		\item \textsl{Vervielfältigung:} Eine ganze Vervielfachung eines Polytop 
		$P$, mit einer natürlichen Zahl $n$, ist ein Polytop $Q$, 
		das sich in $n$ viele Polytope zerlegen lässt, die alle kongruent zu 
		$P$ sind. Wir schreiben für das Polytop $Q$ auch $n\cdot P$. Formal 
		bedeutet dies, dass es Polytope $P_1,\ldots,P_n$ gibt, sd.
		\[n\cdot P=P_1+\ldots+P_n,\]
		wobei $P_i\cong P$ für alle $i=1,\ldots,n$. \\
		Wir bemerken, dass die Lage der Polyeder $P_i$ nicht eindeutig festgelegt 
		ist, was jedoch beim eigentlichen zerlegen keine Rolle spielt.
	\end{itemize}
	
	Es ergeben sich beim Vervielfältigen, wie man leicht sieht folgende Eigenschaften.
	\begin{corollary} \label{coroll:vervielfältigung}
		Seien $P$ und $Q$ zwei Polytope und $n$ eine natürliche Zahl, dann gilt
		\[n\cdot(P+Q)\sim n\cdot P+n\cdot Q.\]
	\end{corollary}
	
	\begin{proof}
		Wir finden also $n$ viele Polytope $R_1,\ldots,R_n$ mit 
		$R_i\cong (P+Q)$ für alle $i=1,\ldots,n$, sd. 
		$n\cdot(P+Q)=R_1+\ldots+R_n$. Das heißt aber auch, dass sich jedes 
		$R_i$ darstellen lässt durch Polytope $P_i$ und $Q_i$ für alle 
		$i=1,\ldots,n$, wobei $P_i\cong P$ und $Q_i\cong Q$. Ebenso finden wir 
		aber auch jeweils $n$ viele Polytope 
		$P_1',\ldots,P_n'$ und $Q_1',\ldots,Q_n'$ mit $P_i'\cong P$ und 
		$Q_i'\cong Q$ für alle $i=1,\ldots,n$, sd. $n\cdot P=P_1'+\ldots+P_n'$ und 
		$n\cdot Q=Q_1'+\ldots+Q_n'$. Mit der Transitivität der Kongruenz 
		gilt also auch $P_i\cong P_i'$ und $Q_i\cong Q_i'$ für alle 
		$i=1,\ldots,n$. Wir erhalten also die Zerlegungen
		\begin{align*}
			n\cdot(P+Q)&=R_1+\ldots+R_n=P_1+Q_1+\ldots+P_n+Q_n \qquad\text{und}\\
			n\cdot P+n\cdot Q&=P_1'+\ldots+P_n'+Q_1'+\ldots+Q_n'=
			P_1'+Q_1'+\ldots+P_n'+Q_n'.
		\end{align*}
		Damit sind $n\cdot (P+Q)$ und $n\cdot P+n\cdot Q$ zerlegungsgleich.
	\end{proof}
	
	\begin{definition}[Minkowski-Summe]
		Seien $A$ und $B$ Mengen, dann ist die Minkowski-Summe $A\times B$ 
		von $A$ und $B$ mit der vektoriellen Addition definiert als
		\[A\times B=\left.\{a+b\ \right\vert\ a\in A,b\in B\}.\]
	\end{definition}
	
	Die Minkowski-Addition ist offensichtlich kommutativ und assoziativ. 
	
	\begin{definition}[$i$-stufiger Zylinder]
		Ein Polytop $P$ ist ein $i$-stufiger Zylinder, falls es eine Darstellung 
		der Form
		\[P=P_1 \times\ldots\times P_i\]
		gibt, wobei $P_i$ hier konvexe Polytope sind, die jeweils in einer $d_i$-dimensionalen Ebene $E_{d_i}$ liegen. Hierbei soll $\sum_1^i d_i =d$
		gelten und die Ebenen $E_{d_i}$ befinden sich paarweise in allgemeiner Lage.
	\end{definition}
	
	\begin{definition}[Zylinderklassen]
		Ein Polytop $P$ ist ein $i$-stufiges Zylinderpolytop, falls 
		es endlich viele $i$-stufige Polytope $P_1,\ldots,P_n$ gibt, sd.
		\[P=P_1+\ldots+P_n.\]
		Wir wollen mit der Zylinderklasse $\mathfrak{B}_i$ die Menge aller 
		Polytope, die zu einem $i$-stufigen Zylinderpolytop zerlegungsgleich 
		sind, wobei $\emptyset\in\mathfrak{B}_i$. Die Stufenzahl $i$, 
		mit $1\leq i\leq  d$, ist die Ordnung der Zylinderklasse $\mathfrak{B}_i$.
	\end{definition}

	Besonders interessant sind hierbei die erste und die $d$-te Zylinderklasse. 
	Es lässt sich nämlich feststellen, dass die erste Zylinderklasse gerade 
	der Menge aller Polytope entspricht. Jedes Polytop lässt sich nach 
	
	%... Jedes Polytop lässt sich in endlich viele Simplizes zerlegen
	
	in endlich viele Simplizes zerlegen und jedes Simplex ist ein $1$-stufiger 
	Zylinder.
	
	%... Aussage oben drüber
	
	Die $d$te Zylinderklasse entspricht gerade Polytopen, welche mit einem 
	Parallelotop zerlegungsgleich sind. Es gilt sogar die folgende Eigenschaft.
	
	\begin{remark} \label{bem:dteZylinderklasse}
		Sei $P\in\mathfrak{B}_d$ ein Polytop, dann gibt es ein $\lambda>0$, sd. 
		$P$ zerlegungsgleich mit $\lambda W_d$ ist. $W_d$ ist hierbei 
		der $d$-dimensionale Einheitswürfel.
		
		%... evtl Begründung über Parallelotope siehe Hadwiger Formel (72)
		
	\end{remark}
	
	Wir stellen fest, dass jeder $i$-stufige Zylinder auch ein $j$-stufiger 
	Zylinder für ein $j<i$ ist und damit erhalten wir
	\[\mathfrak{B}_1\supset \mathfrak{B}_2\supset \ldots\supset\mathfrak{B}_d.\]
	Die Zylinderklassen sind vor allem deswegen für uns interessant, da sie 
	Beweisverfahren nach dem Prinzip der vollständigen Induktion ermöglichen. 
	So können wir Aussagen für alle Polytope mittels Induktion über die 
	Zylinderklassen beweisen. Wir werden dieses Verfahren beim Beweis zu Satz 
	\ref{thm:zerlerg} verwenden.

	\subsection{Ergänzungsgleich und Zerlegungsgleichheit}
	
	Wir wollen noch einmal kurz die Definition der Zerlegungsgleichheit wiederholen.
	
	\begin{definition}[Zerlegungsgleichheit]
		Zwei $d$-Polytope $P$ und $Q$ heißen \textsl{zerlegungsgleich}, wenn es endlich viele $d$-Polytope 
		$P_1,\ldots,P_n,Q_1,\ldots,Q_n$ mit $P=P_1 +\ldots +P_n$,  $Q=Q_1 +\ldots+Q_n$ 
		gibt, sd. 
		\[P_i\cong Q_i\]
		für alle $i\in\{1,\ldots,n\}$. Wir schreiben $P\sim Q$.
	\end{definition}
	
	Nun können wir die Ergänzungsgleichheit definieren.
	
	\begin{definition}[Ergänzungsgleich]
		Zwei $d$-Polytope $P$ und $Q$ heißen \textsl{ergänzungsgleich}, wenn es endlich viele $d$-Polytope 
		$P_1,\ldots,P_n$, $Q_1,\ldots,Q_n$ gibt, wobei gilt $P_i\cong Q_i$ für alle $i\in\{1,\ldots,n\}$, sd. die Polytope
		\[P'=P+P_1+\ldots+P_n,\qquad Q'=Q+Q_1+\ldots+Q_n\]
		zerlegungsgleich sind.
	\end{definition}
	
	Alternativ ist auch die schreibweise üblich, dass zwei Poltope $P$ und $Q$
	ergänzungsgleich sind, falls es zwei zerlegungsgleiche Polytope 
	$A$ und $B$ gibt, sd. $P+A\sim Q+B$ gilt. \\
	Man sieht leicht, dass die Folgende Aussage gilt.
	
	\begin{proposition}\label{coroll:ergvol}
		Seien $P$ und $Q$ zwei ergänzungsgleiche Polytope, dann gilt $vol(P)=vol(Q)$.
	\end{proposition}
	
	\begin{proof}
		Seien $P$ und $Q$ ergänzungsgleich, d. h. es gibt endlich viele Polytope $P_1,\ldots,P_n,
		Q_1\ldots,Q_n$ mit $P_i\cong Q_i$ für alle $i\in{1,\ldots,n}$, sd. 
		$P'=P+P_1+\ldots+P_n$ und $Q'=Q+Q_1+\ldots+Q_n$ zerlegungsgleich sind. Nach 
		Proposition \ref{prop:zerl,vol} gilt dann also auch $vol(P')=vol(Q')$. Damit folgt
		\[vol(P)+vol(P_1)+\ldots+vol(P_n)=vol(P')=vol(Q')=vol(Q)+vol(Q_1)+\ldots+vol(Q_n)\]
		und da mit $P_i\cong Q_i$ für alle $i\in{1,\ldots,n}$ nach Proposition \ref{prop:cong,vol} 
		auch $vol(P_i)=vol(Q_i)$ gilt, folgt $vol(P)=vol(Q)$.
	\end{proof}
	
	\begin{theorem} \label{thm:zerlerg}
		Zwei Polyeder $P$ und $Q$ sind genau dann zerlegungsgleich, wenn sie 
		ergänzungsgleich sind.
	\end{theorem}
	
	Der Beweis richtet sich nach \cite[Satz III]{Hadwiger}.
	
	\begin{proof}
		\noindent
		\begin{itemize}
			\item[$"\Rightarrow"$:] Haben wir zwei zerlegungsgleiche Polytope, so müssen wir kein weiteres Polytop ergänzen, damit die Ergänzungen zerlegungsgleich sind, also folgt bereits die Ergänzungsgleichheit.
			
			\item[$"\Leftarrow"$:] Die Rückrichtung funktioniert per 
			Induktion über die Zylinderklassen. Seien also Polytope 
			$P,Q,A,B$ gegeben, sd. 
			\begin{align}
				P+A\sim Q+B \label{thm:zerlerg;1}
			\end{align}
			und
			\begin{align}
				A\sim B. \label{thm:zerlerg;2}
			\end{align}
			Falls nun $P,Q\in \mathfrak{B}_d$ gilt, dann gibt es mit Bemerkung 
			\ref{bem:dteZylinderklasse} 
			$\lambda_1,\lambda_2 >0$, sd. $P\sim \lambda_1 W_d$ und 
			$Q\sim \lambda_2 W_d$, wobei mit hier $W_d$ der $d$-dimensionale 
			Einheitswürfel gemeint ist. Da mit Folgerung \ref{coroll:ergvol} 
			folgt, dass $vol(P)=vol(Q)$, muss $\lambda_1 =\lambda_2$ und 
			damit $P\sim Q$. \\
			Wir nehmen nun also induktiv an, dass die Aussage bereits für alle 
			Polytope $P,Q\in\mathfrak{B}_{i+1}$ gilt. Es sei nun also 
			$P,Q\in\mathfrak{B}_i$ für ein beliebiges $1\leq i\leq d-1$. 
			Da $d-i>0$ und o.B.d.A. $vol(A)>0$ gibt es eine natürliche Zahl $n$, sd. 
			\begin{align}
				n^{d-i}>1+\frac{vol(A)}{vol(P)}. \label{thm:zerlerg;3}
			\end{align}
			Weiterhin gilt mit \ref{thm:zerlerg;1} und 
			
			%... n(A+B)=nA+nB und aus A sim B folgt nA sim nB
			
			\begin{align}
				nP+nA\sim nQ+nB \label{thm:zerlerg;4}
			\end{align}
			und
			\begin{align}
				nA\sim nB. \label{thm:zerlerg;5}
			\end{align}
			Mit 
			
			%... Hilfssatz I
			
			gibt es nun ein Polytop $P_0\in\mathfrak{B}_{i+1}$, sd.
			\begin{align}
				nP\sim P_0 +n^i\cdot P \label{thm:zerlerg;6}
			\end{align}
			und mit
			
			%... aus zerlegungsgleich folgt volumen gleich
			
			folgt dann
			\begin{align*}
				vol(nP)=vol(P_0+n^i\cdot P)=vol(P_0)+vol(n^i\cdot P)=vol(P_0)+
				n^i\cdot vol(P).
			\end{align*}
			Nach 
			
			%... Volumen bei Dilation für d-dim. Polytop: vol(nP)=n^d vol(P)
			
			gilt $vol(nP)=n^d vol(P)$ und damit 
			\begin{align}
				vol(P_0)=(n^d -n^i)vol(P). \label{thm:zerlerg;7}
			\end{align}
			Weiter folgt mit der Ungleichung \ref{thm:zerlerg;3}, dass 
			$n^d > n^i+n^i\frac{vol(A)}{vol(P)}$ und damit
			\begin{align}
				vol(P_0)=(n^d -n^i)vol(P)>n^i vol(A)=vol(n^i\cdot A) \label{thm:zerlerg;8}.
			\end{align}
			Mit
			
			%... Formel (37)
			
			ist $n^i\cdot A$ nun also zerlegungsgleich zu einem Teilpolytop von $P_0$, dh. es gibt Polytope $C,D\subset P_0$ mit $P+Q=P_0$ und $n^i\cdot A\sim C$, sd. 
			\begin{align}
				P_0=C+D\sim n^i\cdot A+D. \label{thm:zerlerg;9}
			\end{align}
			Damit gilt 
			\begin{align}
				nP &\overset{\ref{thm:zerlerg;9}}{\sim} P_0+n^i \cdot P \notag\\
				&\sim n^iA +D+n^i \cdot P \notag\\
				&\overset{\ref{coroll:vervielfältigung}}{\sim} D+n^i\cdot(A+P) \notag\\ 
				&\overset{\ref{thm:zerlerg;1}}{\sim} D+n^i\cdot(Q+B) \notag\\
				&\overset{\ref{thm:zerlerg;2}}{\sim} D+n^i\cdot(Q+A) \notag\\
				&\overset{\ref{coroll:vervielfältigung}}{\sim}
				Q+n^i\cdot Q +n^i\cdot A \notag\\
				&\overset{\ref{thm:zerlerg;9}}{\sim} P_0+n^i \cdot Q. \label{thm:zerlerg;10}
			\end{align}
			Analog zu \ref{thm:zerlerg;6} finden wir für $Q$ ebenso ein 
			$Q_0\in\mathfrak{B}_{i+1}$ mit $nQ\sim Q_0+n^i\cdot Q$. Wir erhalten 
			\[P_0+n^i\cdot Q+nA\overset{\ref{thm:zerlerg;10}}{\sim}
			nP+nA\overset{\ref{thm:zerlerg;4}}{\sim}nQ+nB\sim Q_0+n^i\cdot Q+nB.\]
			Wegen \ref{thm:zerlerg;5} gilt $n^i\cdot Q+nA\sim n^i\cdot Q +nB$ und 
			mit der Induktionsvoraussetzung folgt für $P_0,Q_0\mathfrak{B}$, dass 
			$P_0\sim Q_0$ und damit folgt mit \ref{thm:zerlerg;10} und der 
			analogen Folgerung, dass 
			\[nP\sim nQ\]
			gilt. Mit
			
			%... aus nP sim nQ folgt P sim Q. Folgerung bei Dilation
			
			gilt also \[P\sim Q.\]
		\end{itemize}
	\end{proof}
	
	
	\newpage
	\bibliographystyle{plain}
	\bibliography{MeineBib}
\end{document}