\documentclass[11pt,titlepage]{article}
\usepackage{amsmath,amssymb,amstext,mathtools,amsthm}
\usepackage{xcolor}
\usepackage[utf8]{inputenc}
\usepackage[ngerman]{babel}
%\usepackage[paper=a4paper,left=25mm,right=25mm,top=25mm,bottom=25mm]{geometry}
\usepackage[twoside,lmargin=3.5cm,rmargin=2.5cm]{geometry}
\usepackage{hyperref}
\hypersetup{bookmarksnumbered}

\usepackage{dsfont}
%\usepackage{xfrac}
\usepackage{tikz}
\usepackage{graphicx}
\usepackage{bigints}
\usepackage{bibgerm}
\usepackage[onehalfspacing]{setspace}

\usetikzlibrary{positioning}
%\usetikzlibrary{arrows}

\newcommand{\setN}{\mathbb{N}}
\newcommand{\setZ}{\mathbb{Z}}
\newcommand{\setQ}{\mathbb{Q}}
\newcommand{\setR}{\mathbb{R}}
\newcommand{\setC}{\mathbb{C}}
\newcommand{\setH}{\mathbb{H}}
\newcommand{\setI}{\mathbb{I}}
\newcommand{\abs}[1]{{\left| #1 \right|}}

\theoremstyle{definition}
\newtheorem{theorem}{Satz}[section]
\newtheorem{corollary}[theorem]{Folgerung}
\newtheorem{proposition}[theorem]{Proposition}
\newtheorem{lemma}[theorem]{Lemma}
\newtheorem{definition}[theorem]{Definition}
\newtheorem{example}[theorem]{Beispiel}
\newtheorem*{axiom}{Axiom}
\newtheorem{remark}[theorem]{Bemerkung}

\theoremstyle{remark}
\newtheorem*{repetition}{Wiederholung}
\newtheorem*{remind}{Erinnerung}

\begin{document}
	\section{Ergänzungsgleichheit von Polytopen}
	
	Wir wollen uns nun dem zweiten Teil von Hilberts Problem widmen und zwar der 
	Ergänzungsgleichheit von Polyedern. Wir wollen in diesem Kapitel zeigen, dass 
	Zerlegungsgleichheit und Ergänzungsgleichheit gleichbedeutend sind. 
	Im Folgenden meinen wir mit Polytopen wieder $d$-Polytope.
	
	Wir wollen noch einmal kurz die Definition der Zerlegungsgleichheit wiederholen.
	
	\begin{definition}[Zerlegungsgleichheit]
		Zwei $d$-Polytope $P$ und $Q$ heißen \textsl{zerlegungsgleich}, wenn es endlich viele $d$-Polytope 
		$P_1,\ldots,P_n,Q_1,\ldots,Q_n$ mit $P=P_1 +\ldots +P_n$,  $Q=Q_1 +\ldots+Q_n$ 
		gibt, sd. 
		\[P_i\cong Q_i\]
		für alle $i\in\{1,\ldots,n\}$. Wir schreiben $P\sim Q$.
	\end{definition}
	
	Nun können wir die Ergänzungsgleichheit definieren.
	
	\begin{definition}[Ergänzungsgleich]
		Zwei $d$-Polytope $P$ und $Q$ heißen \textsl{ergänzungsgleich}, wenn es endlich viele $d$-Polytope 
		$P_1,\ldots,P_n$, $Q_1,\ldots,Q_n$ gibt, wobei gilt $P_i\cong Q_i$ für alle $i\in\{1,\ldots,n\}$, sd. die Polytope
		\[P'=P+P_1+\ldots+P_n,\qquad Q'=Q+Q_1+\ldots+Q_n\]
		zerlegungsgleich sind.
	\end{definition}
	
	Alternativ ist auch die schreibweise üblich, dass zwei Poltope $P$ und $Q$
	ergänzungsgleich sind, falls es zwei zerlegungsgleiche Polytope 
	$A$ und $B$ gibt, sd. $P+A\sim Q+B$ gilt. 
	
	
	\begin{theorem}
		Zwei Polyeder $P$ und $Q$ sind genau dann zerlegungsgleich, wenn sie 
		ergänzungsgleich sind.
	\end{theorem}
	
	Der Beweis richtet sich nach \cite[Satz III]{Hadwiger}.
	
	\begin{proof}
		\noindent
		\begin{itemize}
			\item[$"\Rightarrow"$:] Haben wir zwei zerlegungsgleiche Polytope, so müssen wir kein weiteres Polytop ergänzen, damit die Ergänzungen zerlegungsgleich sind, also folgt bereits die Ergänzungsgleichheit.
			
			\item[$"\Leftarrow"$:] Die Rückrichtung funktioniert per 
			Induktion über die Zylinderklassen. Seien also Polytope 
			$P,Q,A,B$ gegeben, sd. 
			\begin{align}
				P+A\sim Q+B \label{thm:zerlerg;1}
			\end{align}
			und
			\begin{align}
				A\sim B. \label{thm:zerlerg;2}
			\end{align}
			Falls nun $P,Q\in \mathfrak{B}_d$ gilt, dann gibt es mit 
			
			%... Formel (72) Hadwiger
			
			$\lambda_1,\lambda_2 >0$, sd. $P\sim \lambda_1 W_d$ und 
			$Q\sim \lambda_2 W_d$, wobei mit hier $W_d$ der $d$-dimensionale 
			Einheitswürfel gemeint ist. Da mit 
			
			%... Aus ergänzungsgleich folgt gleiches Volumen
			
			folgt, dass $vol(P)=vol(Q)$, muss $\lambda_1 =\lambda_2$ und 
			damit $P\sim Q$. \\
			Wir nehmen nun also induktiv an, dass die Aussage bereits für alle 
			Polytope $P,Q\in\mathfrak{B}_{i+1}$ gilt. Es sei nun also 
			$P,Q\in\mathfrak{B}_i$ für ein beliebiges $1\leq i\leq d-1$. 
			Da $d-i>0$ und o.B.d.A. $vol(A)>0$ gibt es eine natürliche Zahl $n$, sd. 
			\begin{align}
				n^{d-i}>1+\frac{vol(A)}{vol(P)}. \label{thm:zerlerg;3}
			\end{align}
			Weiterhin gilt mit \ref{thm:zerlerg;1} und 
			
			%... n(A+B)=nA+nB und aus A sim B folgt nA sim nB
			
			\begin{align}
				nP+nA\sim nQ+nB \label{thm:zerlerg;4}
			\end{align}
			und
			\begin{align}
				nA\sim nB. \label{thm:zerlerg;5}
			\end{align}
			Mit 
			
			%... Hilfssatz I
			
			gibt es nun ein Polytop $P_0\in\mathfrak{B}_{i+1}$, sd.
			\begin{align}
				nP\sim P_0 +n^i\cdot P \label{thm:zerlerg;6}
			\end{align}
			und mit
			
			%... aus zerlegungsgleich folgt volumen gleich
			
			folgt dann
			\begin{align*}
				vol(nA)=vol(P_0+n^i\cdot P)=vol(P_0)+vol(n^i\cdot P)=vol(P_0)+
				n^i\cdot vol(P).
			\end{align*}
			Nach 
			
			%... Volumen bei Dilation für d-dim. Polytop: vol(nP)=n^d vol(P)
			
			gilt $vol(nP)=n^d vol(P)$ und damit 
			\begin{align}
				vol(P_0)=(n^d -n^i)vol(P). \label{thm:zerlerg;7}
			\end{align}
		\end{itemize}
	\end{proof}
	
	
	\newpage
	\bibliographystyle{plain}
	\bibliography{MeineBib}
\end{document}